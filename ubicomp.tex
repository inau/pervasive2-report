\documentclass{sigchi}

% Arabic page numbers for submission. 
% Remove this line to eliminate page numbers for the camera ready copy
%\pagenumbering{arabic}


% Load basic packages
\usepackage{balance}  % to better equalize the last page
\usepackage{graphics} % for EPS, load graphicx instead
\usepackage{times}    % comment if you want LaTeX's default font
\usepackage{url}      % llt: nicely formatted URLs

% llt: Define a global style for URLs, rather that the default one
\makeatletter
\def\url@leostyle{%
  \@ifundefined{selectfont}{\def\UrlFont{\sf}}{\def\UrlFont{\small\bf\ttfamily}}}
\makeatother
\urlstyle{leo}


% To make various LaTeX processors do the right thing with page size.
\def\pprw{8.5in}
\def\pprh{11in}
\special{papersize=\pprw,\pprh}
\setlength{\paperwidth}{\pprw}
\setlength{\paperheight}{\pprh}
\setlength{\pdfpagewidth}{\pprw}
\setlength{\pdfpageheight}{\pprh}

% Make sure hyperref comes last of your loaded packages, 
% to give it a fighting chance of not being over-written, 
% since its job is to redefine many LaTeX commands.
\usepackage[pdftex]{hyperref}
\hypersetup{
pdftitle={SIGCHI Conference Proceedings Format},
pdfauthor={LaTeX},
pdfkeywords={SIGCHI, proceedings, archival format},
bookmarksnumbered,
pdfstartview={FitH},
colorlinks,
citecolor=black,
filecolor=black,
linkcolor=black,
urlcolor=black,
breaklinks=true,
}

% create a shortcut to typeset table headings
\newcommand\tabhead[1]{\small\textbf{#1}}


% End of preamble. Here it comes the document.
\begin{document}

\title{Situated Display in Hospital Ward}

\numberofauthors{3}
\author{
  \alignauthor Ivan Naumovski\\
    \affaddr{ITU}\\
    \affaddr{Rued Langaardsvej 7, 2300 Copenhagen}\\
    \email{inau@itu.dk}
  \alignauthor Martino Secchi\\
    \affaddr{ITU}\\
    \affaddr{Rued Langaardsvej 7, 2300 Copenhagen}\\
    \email{msec@itu.dk}
  \alignauthor Diem Hong\\
    \affaddr{Affiliation}\\
    \affaddr{Rued Langaardsvej 7, 2300 Copenhagen}\\
    \email{hong@itu.dk}
}

\maketitle

\begin{abstract}
\textit{Etnographic studies of surgeons and their workflows show that issues exists.
This journal attempts to solve some of the issues discovered in these studies.
A system that supports the workflow namely avoiding the spread of bacteria via touchless interaction has been designed.
To build the system multiple areas have been investigated, areas such as hardware prototyping, machine learning and software development.
This resulted in a system with 4 tiers, one handling data collection, another handling communication, a third one doing pattern recognition and the last one for presentation.
The system has quite a high recognition rate, due to some of the design choices taken.
For detailed information refer to the respective areas within the journal.
}
\end{abstract}

\keywords{
	Guides; instructions; author's kit; conference publications;
	keywords should be separated by a semi-colon.
	\textcolor{red}{Mandatory section to be included in your final version.}
}

\category{H.5.2.}{Information Interfaces and Presentation}{Input devices and strategies}
\category{I.2.6.}{Artificial Intelligence}{Learning}

See: \url{http://www.acm.org/about/class/1998/}
for more information and the full list of ACM classifiers
and descriptors. 

\section{Introduction}
\section{Introduction}
Interacting with machines has always been a hot topic in the field of Ubicomp.
Articles such as 

Modalities have been an important aspect
to consider when interfacing with the machines, whether it has been done using keystokes, audio-control or gestures.

It has always been of high relevance to push the limits for how we interact with technology, mainly to empower the individual.

Technology is supposed to make common tasks even easier to achieve. Consider the case of a physically disabled person being empowered by using an alternative modality such as voice control.
%We have pushed the boundaries for how interaction with a computer is done.

In our case the focus is to improve on how people in the healthcare sector interact with technology.
In the healthcare sector interaction can actually prove to have negative effects, particularly bacteria can be transferred.
For example, in most operating room at hospital, large displays and computers capture medical information during surgery. Because of sterility, the input devices are not handled directly by surgeon but by assistants or nurses who may cause some misunderstandings and delays\cite{Pederson:2015}. The wearable devices are suggested to resolve this problems.

One way to avoid spreading bacteria is by avoiding physical interaction with input devices.

This can result in less time spent scrubbing or rinsing as the likelihood for microbacteria being 
transfered via input devices is reduced.


Ultimately alternative modalities could provide employees within the sector a 
way to use native hand gestures to perform tasks such as interacting with x-ray images.


\section{Previous Work}
\todo{Background}
There exist alot of prior work within the field of HCI which is relevant to our solution.
The papers which influenced this project the most will be mentioned in following section.

\subsubsection{Alternative Modalities}
Prior work in regards to pattern recognition has been of high relevance to our solution.
bla bla bla bla

\subsubsection{Smart Machines}
Prior work in regards to pattern recognition has been of high relevance to our solution.
bla bla bla bla



\section{Our solution - better heading}
\section{System Architecture}
The system we have designed consists of two parts,
one being a input device
and the other being an application able to recognize the patterns from the device and forward predefined commands.

\subsection{The Input Device}
Our motion tracking device has been built in such a fashion that it resembles a wrist watch.
This form factor makes it rather compact.
Additionally a lot of people wear watches on a daily basis which makes it recognizable and barely noticeable for people.

The specific device contains a list of components, the most important are 6DoF Sensor, Bluetooth and Arduino microcontroller.

\textbf{6DoF Sensor:}\\
This is the heart of the device. It contains an accelerometer and a gyroscope. This means we can measure movement and rotation of the wrist of anyone wearing the device.

\textbf{Arduino Pro Mini:}\\
This is the brain of the device, it handles all communication between the 6DoF chip and forwards the data over bluetooth to any consumers of the data.

The others are not that important to spend a lot of time on, they provide means of communicating, charging and toggling the device on and off.

\begin{figure}[!h]
\centering
\includegraphics[width=0.9\columnwidth]{img/device_schematic}
\caption{This schematic shows the circuitry on the device.}
\label{fig:figure1}
\end{figure}

\subsection{Weka Gesture Recognition System}
Since the input device we have built is quite lightweight, data processing needs to be done elsewhere.
This provided us with the challenge of transferring data from one bluetooth capable device to another.
All the processing and preprocessing of the device's data is performed on a desktop application on a 
computer connected to the device.
Especially challenging was to find and use Java libraries that would allow to establish a Bluetooth 
connection with the device and to exchange data with the desktop application that would process it. 
For further technical details about the challenges encountered please refer to the Discussion section. 
\todo{add technical details about the rxtx fucking everything up}
Once the acceleration and rotation data is sent from the device to the computer,
 the values are smoothed with an average of the 20 previous values in order to avoid and reduce the effect of noise on the sensors.
 This phase is called preprocessing of the data. 
 This allowed us to have more precise information, and switch from this:

\begin{figure}[!h]
\centering
\includegraphics[width=0.9\columnwidth]{img/raw}
\caption{Data from the device before preprocessing.}
\label{fig:figure2}
\end{figure}

To this:

\begin{figure}[!h]
\centering
\includegraphics[width=0.9\columnwidth]{img/20}
\caption{Data from the device after preprocessing.}
\label{fig:figure3}
\end{figure}

For the actual processing and gesture recognition, we organized known gestures in a large training set, each individual gesture stored as a list of 50 * 6 values plus an identifier. 
We use Weka 3.6 to evaluate newly received data using a BayesNet classifier and comparing it to the training set.
The evaluation of the gesture is performed every 10 * 6 new values received.

\subsection{Android Application}

\subsection{Webservice}


\subsubsection{Pattern Recognition}

\section{Result}
\section{Result}
The system sometimes encounters bluetooth interference from other devices which
cause the system not to work properly. Moreover, the weak Wi-Fi signals made system delays when simulating the hand gestures. 


\section{Discussion}

One of the less performant aspects of the project was the bluetooth connection, which was very unstable. 
We account interference in the ITU building mostly responsible for that. \todo{other reasons?}

\pagebreak
\begin{figure}[!h]
\centering
\includegraphics[width=0.9\columnwidth]{Figure1}
\caption{With Caption Below, be sure to have a good resolution image
  (see item D within the preparation instructions).}
\label{fig:figure1}
\end{figure}

\subsection{References and Citations}
Use a numbered list of references at the end of the article, ordered
alphabetically by first author, and referenced by numbers in brackets
\cite{ethics,
  Klemmer:2002:WSC:503376.503378,
  Mather:2000:MUT,
  Zellweger:2001:FAO:504216.504224}. For
papers from conference proceedings, include the title of the paper and
an abbreviated name of the conference (e.g., for Interact 2003
proceedings, use \textit{Proc. Interact 2003}). Do not include the
location of the conference or the exact date; do include the page
numbers if available. See the examples of citations at the end of this
document. Within this template file, use the \texttt{References} style
for the text of your citation.

Your references should be published materials accessible to the
public.  Internal technical reports may be cited only if they are
easily accessible (i.e., you provide the address for obtaining the
report within your citation) and may be obtained by any reader for a
nominal fee.  Proprietary information may not be cited. Private
communications should be acknowledged in the main text, not referenced
(e.g., ``[Robertson, personal communication]'').

\begin{table}
  \centering
  \begin{tabular}{|c|c|c|}
    \hline
    \tabhead{Objects} &
    \multicolumn{1}{|p{0.3\columnwidth}|}{\centering\tabhead{Caption --- pre-2002}} &
    \multicolumn{1}{|p{0.4\columnwidth}|}{\centering\tabhead{Caption --- 2003 and afterwards}} \\
    \hline
    Tables & Above & Below \\
    \hline
    Figures & Below & Below \\
    \hline
  \end{tabular}
  \caption{Table captions should be placed below the table.}
  \label{tab:table1}
\end{table}

\section{Figures/Captions}
Place figures and tables at the top or bottom of the appropriate
column or columns, on the same page as the relevant text
(see Figure~\ref{fig:figure1}). A figure or table may extend across both
columns to a maximum width of 17.78 cm (7 in.).

Captions should be Times New Roman 9-point bold.  They should be numbered (e.g.,
``Table~\ref{tab:table1}'' or ``Figure~\ref{fig:figure2}''), centered
and placed beneath the figure or table.  Please note that the words
``Figure'' and ``Table'' should be spelled out (e.g., ``Figure''
rather than ``Fig.'') wherever they occur.

Papers and notes may use color figures, which are included in the page
limit; the figures must be usable when printed in black and white in
the proceedings.  The paper may be accompanied by a short video figure
up to five minutes in length.  However, the paper should stand on its
own without the video figure, as the video may not be available to
everyone who reads the paper.

\section{Language, Style and Content}

The written and spoken language of SIGCHI is English. Spelling and
punctuation may use any dialect of English (e.g., British, Canadian,
US, etc.) provided this is done consistently. Hyphenation is
optional. To ensure suitability for an international audience, please
pay attention to the following:

\begin{itemize}
\item Write in a straightforward style.
\item Try to avoid long or complex sentence structures.
\item Briefly define or explain all technical terms that may be
  unfamiliar to readers.
\item Explain all acronyms the first time they are used in your text---e.g.,
``Digital Signal Processing (DSP)''.
\item Explain local references (e.g., not everyone knows all city
  names in a particular country).
\item Explain ``insider'' comments. Ensure that your whole audience
  understands any reference whose meaning you do not describe (e.g.,
  do not assume that everyone has used a Macintosh or a particular
  application).
\item Explain colloquial language and puns. Understanding phrases like
  ``red herring'' may require a local knowledge of English.  Humor and
  irony are difficult to translate.
\item Use unambiguous forms for culturally localized concepts, such as
  times, dates, currencies and numbers (e.g., ``1-5-97'' or ``5/1/97''
  may mean 5 January or 1 May, and ``seven o'clock'' may mean 7:00 am or
  19:00).  For currencies, indicate equivalences---e.g., ``Participants
  were paid 10,000 lire, or roughly \$5.''
\item Be careful with the use of gender-specific pronouns (he, she)
  and other gendered words (chairman, manpower, man-months). Use
  inclusive language that is gender-neutral (e.g., she or he, they,
  s/he, chair, staff, staff-hours,
  person-years). See~\cite{Schwartz:1995:GBF} for further advice and
  examples regarding gender and other personal attributes.
\item If possible, use the full (extended) alphabetic character set
  for names of persons, institutions, and places (e.g.,
  Gr{\o}nb{\ae}k, Lafreni\'ere, S\'anchez, Universit{\"a}t,
  Wei{\ss}enbach, Z{\"u}llighoven, \r{A}rhus, etc.).  These characters
  are already included in most versions of Times, Helvetica, and Arial
  fonts.
\end{itemize}

\section{Acknowledgments}
We thank CHI, PDC and CSCW volunteers, and all publications support
and staff, who wrote and provided helpful comments on previous
versions of this document.  Some of the references cited in this paper
are included for illustrative purposes only.  \textbf{Don't forget
to acknowledge funding sources as well}, so you don't wind up
having to correct it later.

% Balancing columns in a ref list is a bit of a pain because you
% either use a hack like flushend or balance, or manually insert
% a column break.  http://www.tex.ac.uk/cgi-bin/texfaq2html?label=balance
% multicols doesn't work because we're already in two-column mode,
% and flushend isn't awesome, so I choose balance.  See this
% for more info: http://cs.brown.edu/system/software/latex/doc/balance.pdf
%
% Note that in a perfect world balance wants to be in the first
% column of the last page.
%
% If balance doesn't work for you, you can remove that and
% hard-code a column break into the bbl file right before you
% submit:
%
% http://stackoverflow.com/questions/2149854/how-to-manually-equalize-columns-
% in-an-ieee-paper-if-using-bibtex
%
% Or, just remove \balance and give up on balancing the last page.
%
\balance

% If you want to use smaller typesetting for the reference list,
% uncomment the following line:
% \small
\bibliographystyle{acm-sigchi}
\bibliography{ubicomp}
\end{document}
