\section{Introduction}
The field of Ubiqutous computing has since its origin been a field which is hard to define due to it constantly changing.
While in its early stages it focussed primarilly on human computer interaction(HCI) 
and trying to reinvent the way end-users interact with machines.
Later on the field covers a lot more, spanning from communication infrastructure to 
An example of such a change in HCI paragdigms can be illustrated by comparing the command-line to the desktop. Keyboard input compared to pointer input [\cite{}].
Modalities have been an important aspect
to consider when interfacing with the machines, whether it has been done using keystokes, audio-control or gestures.

It has always been of high relevance to push the limits for how we interact with technology, mainly to empower the individual.

Technology is supposed to make common tasks even easier to achieve. Consider the case of a physically disabled person being empowered by using an alternative modality such as voice control.
%We have pushed the boundaries for how interaction with a computer is done.

In our case the focus is to improve on how people in the healthcare sector interact with technology.
In the healthcare sector interaction can actually prove to have negative effects, particularly bacteria can be transferred.
By avoiding direct interaction contamination can be avoided.

\todo{is the following a quote or just a deduction from the paper?}
In most operating rooms at hospitals, large displays and computers capture medical information during surgery.
Because of sterility, the input devices are not handled directly by surgeons but by assistants or nurses which may result in misunderstandings or delays\cite{Pederson:2015}.
The wearable devices are suggested to resolve this problems.

One way to avoid spreading bacteria is by avoiding physical interaction with input devices.

This can result in less time spent scrubbing or rinsing as the likelihood for microbacteria being 
transfered via input devices is reduced.


Ultimately alternative modalities could provide employees within the sector a 
way to use native hand gestures to perform tasks such as interacting with x-ray images.
