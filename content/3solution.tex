\section{System Architecture}
The system we have designed consists of multiple parts,
one being a input device, another being a data processer, 
a third one being a communications service and the last one being a presenter.

\subsection{The Input Device}
Our motion tracking device(MTD) has been built in such a fashion that it resembles a wrist watch.
This form factor makes it rather compact.
Additionally a lot of people wear watches on a daily basis which makes the device of a recognizable morphology and barely noticeable for people accustomed to such devices.
The MTD contains a list of components, the most important are 6DoF Sensor, Bluetooth and the Arduino microcontroller.

The \textbf{6DoF Sensor} is the heart of the device.
It is a board which contains an accelerometer and a gyroscope.
This means that it is possible to measure movement and rotation of the wrist of anyone wearing the device.
The sensor produces a set of 6 values: 3 for acceleration and 3 for rotation, each representing information on the axes of the coordinate system.
The input received from the 6DoF sensor is then propagated to the \textbf{arduino microcontroller}.
The arduino prepares the raw  values into JSON data. The JSON is then sent using the bluetooth device.

\begin{figure}[!h]
\centering
\includegraphics[width=0.9\columnwidth]{img/device_schematic}
\caption{This schematic shows the circuitry on the device.}
\label{fig:figure1}
\end{figure}

\subsection{Weka Gesture Recognition System}
Since the input device we have built is quite lightweight, data processing needs to be done elsewhere.
This provided us with the challenge of transferring data from one bluetooth capable device to another.
All the processing and preprocessing of the device's data is performed on a desktop application on a 
computer connected to the device.
Especially challenging was to find and use Java libraries that would allow to establish a Bluetooth 
connection with the device and to exchange data with the desktop application that would process it. 
For further technical details about the challenges encountered please refer to the Discussion section. 
\todo{add technical details about the rxtx fucking everything up}
Once the acceleration and rotation data is sent from the device to the computer,
 the values are smoothed with an average of the 20 previous values in order to avoid and reduce the effect of noise on the sensors.
 This phase is called preprocessing of the data. 
 This allowed us to have more precise information, as can be deducted from \textbf{Figure 2} and \textbf{Figure 3}

\begin{figure}[!h]
\centering
\includegraphics[width=0.9\columnwidth]{img/raw}
\caption{Data from the device before preprocessing.}
\label{fig:figure2}
\end{figure}

\begin{figure}[!h]
\centering
\includegraphics[width=0.9\columnwidth]{img/20}
\caption{Data from the device after preprocessing.}
\label{fig:figure3}
\end{figure}

For the actual processing and gesture recognition, we organized known gestures in a large training set, each individual gesture stored as a list of 50 * 6 values plus an identifier. 
We use Weka 3.6 to evaluate newly received data using a BayesNet classifier and comparing it to the training set.
The evaluation of the gesture is performed every 10 * 6 new values received.

\subsection{Android Application}
The android system is a quite limited system.
It is able to display  an image.
It provides the user with the possibility of panning and zooming a given image.
When the user zooms all the way out panning provides additional functionality.
Pannning left and right will then swap images from the set of loaded images.

\subsection{Webservice}
The communication between the Weka Gesture Recognition System and the Android Application is done by a simple webservice.
It provides GET, POST and DELETE requests which manipulate a queue.

GET pops all the queued gesture recognitions, 
POST pushes a new one to the service and DELETE clears the queue.

\begin{figure}[!h]
\centering
\includegraphics[width=0.9\columnwidth]{img/system_diagram}
\caption{This schematic shows the system interactions.}
\label{fig:figure4}
\end{figure}